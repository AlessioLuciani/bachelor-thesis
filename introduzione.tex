\chapter*{Introduzione}
\addcontentsline{toc}{chapter}{Introduzione}
In questa relazione viene discusso il lavoro che è stato
portato avanti durante il tirocinio e che consiste nello
sviluppo di una serie di funzionalità integrate in un'applicazione
mobile di ambito smart-parking. Più in particolare, le funzionalità
sono mirate alla creazione di un modello machine learning che
sia in grado di classificare diversi tipi di parcheggio (a spina,
a pettine, parallelo) ed utilizzare l'informazione ottenuta 
per diversi scopi. Il tutto è stato reso possibile dall'attività
svolta durante il percorso di eccellenza, in cui è stato creato
un programma in grado di processare i dati raccolti e addestrare il 
modello ML.\\
L'app in questione è GeneroCity (nella sua versione per piattaforma
iOS). L'utilità principale di questa applicazione è quella di 
mettere in contatto degli utenti che stanno cercando un parcheggio
pubblico in una zona urbana con altri che invece ne stanno lasciando
uno. Infatti, l'idea di base sta nel fatto di poter scambiare
"generosamente" e reciprocamente un posto auto, creando un match tra un utente
che cerca parcheggio in una certa zona e un altro che ne lascia uno
nella stessa.\\
Verrà descritto come avviene la formazione del dataset necessario per
l'addestramento del modello, come il contributo dell'utente sia 
fondamentale per questo fine e come infine il modello ottenuto viene
integrato nell'applicazione. Si scenderà nel dettaglio di come sono
state effettuate tutte queste operazioni e si discuterà dei problemi
che si sono riscontrati durante il processo.\\
Successivamente verranno introdotte delle applicazioni, all'interno di 
GeneroCity, che sono state rese possibili dall'ottenimento 
dell'informazione sul tipo di parcheggio e non solo. Un esempio
riguarda la rappresentazione visuale dei parcheggi sulla mappa
dell'app, che mostra informazioni come il tipo, l'orientamento geografico, ecc.\\
Alla fine verranno anticipati degli sviluppi futuri che renderanno
questo lavoro ancora più utile per quanto riguarda le funzionalità
di GeneroCity e l'esperienza che l'utente ne trae.