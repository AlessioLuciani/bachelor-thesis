\chapter{Deploy dei modelli ML}

Una volta che è stato effettuato correttamente il training e la creazione di un modello
machine learning, arriva il momento di distribuirlo e di utilizzarlo effettivamente
per il suo scopo, ovvero quello di classificatore di tipi di parcheggio.\\
L'idea generale del funzionamento consiste nel raccogliere dati durante il tragitto
dell'auto e in seguito fornirli come input al modello. Chiaramente, i dati che 
il modello prodotto riceve in input devono essere della stessa forma di quelli
utilizzati per fare il training. Ciò implica che tutte le procedure di pulizia
dei dati e di miglioramento delle feature debbano essere ripetute ogni volta che
nuovi dati vengono raccolti per essere sottoposti al modello ed essere 
classificati.\\
Garantire che un modello riceva in input dati di forma e qualità identiche a
quelle del dataset di training può talvolta risultare laborioso. Ciò è causato 
dal fatto che gli ambienti di raccolta dati e quelli di produzione possono 
differire sotto svariati aspetti. Infatti, non è scontato che in queste due
situazioni si utilizzino lo stesso linguaggio di programmazione, lo stesso sistema
operativo, le stesse librerie, gli stessi approcci, ecc... Al contrario, è
molto probabile che un modello venga istruito una volta, per poi essere 
distribuito ed utilizzato su piattaforme diverse. In queste situazioni si può
andare incontro a numerose complicazioni, come dover ri-progettare alcuni
algoritmi a causa di un cambiamento di linguaggio di programmazione, che 
renderebbe l'algoritmo stesso meno efficente o addirittura non più funzionante.
In questo caso, è fondamentale avere una conoscenza approfondita del funzionamento
dell'algoritmo originale e della buona documentazione da poter consultare. 
Qualsiasi minimo errore di trascrizione o di comprensione potrebbe generare
modifiche sostanziali ai dati che verranno processati, rendendoli così differenti
da quelli provenienti dal training set e non più validi per una potenziale
classificazione. Spesso si può anche andare incontro ad un cambiamento di librerie,
soprattutto se si sta cambiando linguaggio di programmazione. La questione che
riguarda le librerie è ancora più delicata rispetto a quella dei linguaggi. Nella
maggior parte dei casi, esse contengono delle logiche interne che risultano molto
complesse da riprodurre o imitare in un ambiente differente. Per questo motivo,
è bene ridurre al minimo l'utilizzo di librerie esterne quando ci si trova in 
situazioni come questa, ovvero scrivendo codice che dovrà essere portato su altre
piattaforme. Tuttavia, a volte è necessario utilizzare alcune librerie, come nel
nostro caso, in cui si sono dovute utilizzare libreria per l'algebra lineare.
Quindi, anche per questo discorso occorre avere massima prudenza nella scelta di
un sostituto valido e che non generi inconsistenze nei dati.
 

\section{Necessità del calcolo in locale}
\subsection{Predizione in locale con CoreML per sfruttare il neural engine e non appesantire
il server...}
\section{Porting del processore di dati}
\section{Adattamento all'ambiente mobile (iOS)}
\subsection{Librerie differenti...}










