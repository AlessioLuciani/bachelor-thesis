\chapter{Contributo dell'utente}

Come è stato già discusso, nell'ambito della raccolta dei dati il contributo dell'utente
è essenziale. Infatti, è bene che quest'ultimo sia veritiero e che esso venga fornito per
il maggior numero di registrazioni possibile. Per agevolare l'utente nel dare il contributo
è stato necessario creare un'interazione semplice e che non richiedesse procedure troppo 
sofisticate.\\
Dato che l'informazione che deve essere fornita dall'utente è una etichetta che classifichi
il tipo di parcheggio appena effettuato, è stato deciso di mostrare una notifica di sistema
che proponesse questa scelta. Questa notifica viene chiaramente manifestata dal sistema
sotto forma di notifica locale di GeneroCity e da modo di selezionare il tipo di parcheggio
effettuato.

\section{Notifica mostrata}

Su piattaforma iOS, è stata scelta la notifica di tipo \emph{Actionable Notification}.
Le notifiche di questa tipologia permettono di aggiungere, oltre a del testo, un gruppo 
di pulsanti, corrispondenti a delle azioni. Questa è risultata una soluzione ottima per 
il nostro scopo. \'E stato potuto aggiungere del testo che chiedesse "In che modo hai
parcheggiato?" e allegare dei pulsanti con le opzioni "A spina", "Parallelo", "A pettine".
In questo modo, all'utente resta solo da rispondere alla notifica selezionando una 
opzione tra quelle disponibili.

\subsection{Registrazione della notifica}

Per creare la nuova notifica, la prima cosa che è stato necessario fare è stata registrarla
nel sistema, attraverso l'app. Dato che si tratta di una \emph{Actionable Notification},
è stata registrata una \emph{UNNotificationCategory} e delle \emph{UNNotificationAction}
legate ad essa. Queste classi provengono tutte dalla libreria \emph{UserNotifications}. 
La categoria è necessaria per distinguere una \emph{Actionable Notification}
di un tipo da una di un altro. Quindi, è stata creata una nuova categoria chiamata 
\emph{PARKTYPE}, utilizzata esclusivamente per questa notifica. A questa categoria sono 
state legate le varie azioni, che corrispondono ai tipi di parcheggio. Anche esse sono
dotate di un identificatore che è utilizzato al momento della ricezione, per distinguerle
tra tutte le azioni che l'app può ricevere.

\subsection{Esposizione della notifica}

Ogni volta che una nuova registrazione termina, viene creata e mostrata una nuova notifica
con categoria \emph{PARKTYPE}. Per fare ciò, viene creato un
\emph{UNTimeIntervalNotificationTrigger} con un ritardo di 5 secondi. Questo significa che
la notifica verrà effettivamente mostrata all'utente 5 secondi dopo che GeneroCity abbia
rilevato il parcheggio effettuato dall'utente. In questo modo, esso può rispondere alla
notifica quando effettivamente è a conoscenza del tipo di parcheggio appena fatto e
inoltre si è sicuri che non sia più alla guida e quindi non corra rischi perdendo l'
attenzione. % TODO: maybe link here paper "Driver Distraction from Dashboard and Wearable Interfaces"

\section{Etichetta selezionata dall'utente}

Nel momento in cui l'utente riceve la notifica, esso può rispondere scegliendo un'opzione,
oppure ignorarla completamente. Nel caso in cui la notifica viene ignorata, semplicemente 
non accade nulla, ovvero, l'etichetta del tipo di parcheggio non viene caricata sul
database. Invece, in caso contrario, è necessario aggiornare il valore nel database per 
legare l'etichetta appena ottenuta attraverso l'utente, al relativo parcheggio.

\subsection{Ricezione della risposta}

Attraverso il \emph{UNUserNotificationCenterDelegate}, si possono registrare callback 
da eseguire in seguito ad alcuni eventi generati dal sistema, che riguardano le notifiche.
Uno di questi eventi è la ricezione di una risposta. Questa callback viene invocata quando
l'utente clicca su un pulsante azione di una qualsiasi notifica legata all'app.
Questo significa che è compito dello sviluppatore individuare quale azione è stata eseguita
dall'utente, all'interno della callback. Ciò viene fatto attraverso gli identificatori
delle \emph{UNNotificationAction}. Una volta che è stata identificata l'azione scelta,
è possibile capire qual'è il tipo di parcheggio effettuato dall'utente e salvarlo
all'interno dell'oggetto \emph{Car}. Per terminare, viene effettuata una chiamata all'
API del backend, che aggiorni i dati del parcheggio relativo all'auto corrente
nel database.