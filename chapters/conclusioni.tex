\chapter{Conclusioni}

\section{Sommario}

In questo elaborato è stato descritto il lavoro portato avanti durante
l'attività di tirocinio. In particolare, il filo conduttore è stato
la classificazione del tipo di parcheggio nell'app GeneroCity 
e l'utilizzo di questa informazione per diversi scopi.\\
In un primo momento si è parlato della procedura di raccolta dati,
mirata all'ottenimento di un dataset utilizzabile per effettuare
il training del modello classificatore. Sono stati descritti in
dettaglio i sensori utilizzati e introdotti i tipi di modello ML
scelti.\\
Poi è stato spiegato come l'utente contribuisce nella formazione del
dataset. Ovvero, l'interazione che l'utente ha con la notifica e 
come avviene la selezione dell'etichetta per una specifica registrazione
di campioni.\\
Si è proceduto con la spiegazione di come viene effettuato il deploy
dei modelli addestrati all'interno dell'app. Sono stati descritte tutte 
le azioni che sono risultate necessarie per la predizione in locale, come
il porting del processore di dati da Python a Swift e l'integrazione di 
\emph{Core ML}.\\
Infine si è parlato di come l'informazione sul tipo di parcheggio sia stata
utilizzata a beneficio dell'utente. Così è stata descritta la 
rappresentazione dei parcheggi sulla mappa: il modo di richiedere delle 
informazioni riguardanti i parcheggi visualizzati all'API del backend, le
caratteristiche visuali che gli sono state attribuite, ecc.

\section{Sviluppi futuri}

Si può pensare a diversi sviluppi futuri che possano migliorare l'integrazione
e l'utilità di questo lavoro all'interno dell'app.\\
Innanzitutto, la dimensione del dataset al momento della scrittura è ancora
insufficiente per permettere di ottenere dei modelli abbastanza maturi. Dunque,
come prima cosa si cercherà di coinvolgere un gruppo più grande di utenti
per raccogliere molti altri dati in tempi ragionevoli.\\
Tra le funzionalità descritte vi è quella della rappresentazione dei parcheggi 
sulla mappa. Questa rappresentazione tiene conto di tutte le istanze di parcheggio
effettuate da utenti reali, quindi di ogni singolo parcheggio registrato, non 
considerando l'accuratezza di questi parcheggi. Uno sviluppo che potrebbe essere
fatto riguarda un miglioramento nel salvataggio dei parcheggi nel database. Infatti,
potrebbero venir usate le singole istanze, insieme ad altre informazioni, per
ottenere una stima di dove si trovi veramente un certo posteggio. In questo modo
si otterrebbe un insieme di parcheggi classificati più accuratamente.\\
Un'altra prossima aggiunta che verrà fatta sarà il miglioramento dell'algoritmo
di matching, tenendo conto dell'informazione sul tipo di parcheggio. Cosí sarà 
possibile evitare match il cui scambio risulta impossibile a causa delle dimensioni
diverse delle due auto.